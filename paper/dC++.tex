\documentclass{article}

\usepackage[english]{babel}
\usepackage{amsmath}
\usepackage{amsfonts}
\usepackage{amsthm}
\usepackage{caption}
\usepackage{bbm}
\usepackage[utf8]{inputenc}
\usepackage[T1]{fontenc}
\usepackage{inconsolata}

\newcommand{\RR}{\mathbb{R}}
\newcommand{\Shift}{\mathcal{S}}
\newcommand{\II}{\mathbb{I}}
\newcommand{\JJ}{\mathbb{J}}
\newcommand{\E}{\mathcal{E}}
\newcommand{\T}{\mathcal{T}}
\newcommand{\VV}{\mathcal{V}}
\newcommand{\MM}{\mathcal{M}}
\newcommand{\NN}{\mathcal{N}}
\newcommand{\e}{\mathbf{e}}
\newcommand{\x}{\mathbf{x}}
\newcommand{\m}{\mathbf{m}}
\newcommand{\uu}{\mathbf{u}}
\newcommand{\vv}{\mathbf{v}}
\newcommand{\F}{\mathcal{F}}
\newcommand{\X}{\mathcal{X}}
\newcommand{\CC}{C\nolinebreak\hspace{-.05em}\raisebox{.4ex}{\tiny\bf +}\nolinebreak\hspace{-.10em}\raisebox{.4ex}{\tiny\bf +}}
\def\CC{{C\nolinebreak[4]\hspace{-.05em}\raisebox{.4ex}{\tiny\bf ++}}}
\newcommand{\dP}{\mathcal{P}}
% operator odvoda
\newcommand{\D}{\partial}
%operator 1 + \D
\newcommand{\DD}{\mathcal{D}}
% operator 1+ \D + \D^2 + ...
\newcommand{\sumd}{\tau}
\newcommand{\Op}{\partial^{\bigoplus}}
\newcommand{\op}[1]{\partial^{#1\bigoplus}}
\DeclareMathOperator{\interior}{int}

\usepackage{listings}
\usepackage{xcolor}
\lstset { %
    language=C++,
    backgroundcolor=\color{black!5}, % set backgroundcolor
    basicstyle=\footnotesize,% basic font setting
}

\newtheorem{definicija}{Definition}[section]
\newtheorem{trditev}{Claim}[section]
\newtheorem{izrek}{Theorem}[section]
\newtheorem{opomba}{Remark}[section]
\newtheorem{corollary}{Corollary}[section]
\newtheorem{proposition}{Proposition}[section]

\title{$\DD^n\CC$}
\author{Žiga Sajovic}

\begin{document}

\maketitle

\begin{abstract}
We provide an illustrative implementation of an analytic, infinitely-differentiable programming space constructed in the paper \emph{Operational calculus on programming spaces and generalized tensor networks}. Implementation closely follows theorems and derivations of the paper, intended as an educational guide.
\end{abstract}

\section{Introduction}

We provide an illustrative implementation of the virtual memory $\VV$ and its expansion to $\VV\otimes T(\VV^*)$, serving by itself as an algebra of programs and an infinitely differentiable programming space
\begin{equation}
\DD^n\CC<\dP_n:\VV\to\VV\otimes T(\VV^*)
\end{equation}
acting on it.

We provide a construction of operators
\begin{equation}
\DD^n=\{\D^k;\quad 0\le k\le n\}
\end{equation}
allowing for implementation of the operator
\begin{equation}\label{eq:sumd}
\sumd^n=1+\D+\D^2+\cdots+\D^n
\end{equation}
increasing the order of a differentiable programming space
\begin{equation}
\sumd^n:\dP_k\to\dP_{n+k}
\end{equation}

All required theorems and proofs are provided by the paper \emph{Operational calculus on programming spaces and generalized tensor networks}.

\section{Virtual memory $\VV_n$}

We model an element of the virtual memory $v\in\VV_n$ 
\begin{eqnarray}
\VV_{n}=\VV_{n-1}\oplus(V_{n-1}\otimes\VV^*) \\
\VV_{n}=\VV\oplus\VV\otimes\VV^*\oplus\cdots\oplus\VV\otimes\VV^{*n\otimes}\\ \label{eq:V_n}
\VV_n=\VV\otimes T(\VV^*)
\end{eqnarray}
with the class $var$.

\begin{lstlisting}
template<class V>
class var
{
    public:
    	int order;
        V id;
        std::map<&var,var>* dTau;

        var();
        var(const var& other);
        ~var();
        var(double n);
        void init();
        /*
        *declerations of algebraic operations
        */
};
\end{lstlisting}

The expanded virtual memory $\VV_n$ is the tensor product of the virtual memory $\VV$ with the tensor algebra of its dual. The address \emph{\&var} stands for the component of $v\in\VV_{n-1}$ the tensor product with the component of $v^*\in\VV^*$ was computed on to generate $v\in\VV_n$ in equation \eqref{eq:V_n}. This depth is contained in the \emph{int order}.

Tensor products of the virtual memory $\VV$ with its dual are modeled using maps, denoted by \emph{dTau}. Naming reflects how the algebra will be constructed, mimicking the operator $\sumd^n$ \eqref{eq:sumd}.

\end{document}