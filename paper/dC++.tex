\documentclass{article}

\usepackage[english]{babel}
\usepackage{amsmath}
\usepackage{amsfonts}
\usepackage{amsthm}
\usepackage{caption}
\usepackage{bbm}
\usepackage[utf8]{inputenc}
\usepackage[T1]{fontenc}
\usepackage{inconsolata}

\usepackage[sorting=none,backend=bibtex]{biblatex}
\bibliography{biblio}


\newcommand{\RR}{\mathbb{R}}
\newcommand{\Shift}{\mathcal{S}}
\newcommand{\II}{\mathbb{I}}
\newcommand{\JJ}{\mathbb{J}}
\newcommand{\E}{\mathcal{E}}
\newcommand{\T}{\mathcal{T}}
\newcommand{\VV}{\mathcal{V}}
\newcommand{\MM}{\mathcal{M}}
\newcommand{\NN}{\mathcal{N}}
\newcommand{\e}{\mathbf{e}}
\newcommand{\x}{\mathbf{x}}
\newcommand{\m}{\mathbf{m}}
\newcommand{\uu}{\mathbf{u}}
\newcommand{\vv}{\mathbf{v}}
\newcommand{\F}{\mathcal{F}}
\newcommand{\X}{\mathcal{X}}
\newcommand{\CC}{C\nolinebreak\hspace{-.05em}\raisebox{.4ex}{\tiny\bf +}\nolinebreak\hspace{-.10em}\raisebox{.4ex}{\tiny\bf +}}
\def\CC{{C\nolinebreak[4]\hspace{-.05em}\raisebox{.4ex}{\tiny\bf ++}}}
\newcommand{\dP}{\mathcal{P}}
% operator odvoda
\newcommand{\D}{\partial}
%operator 1 + \D
\newcommand{\DD}{\mathcal{D}}
% operator 1+ \D + \D^2 + ...
\newcommand{\sumd}{\tau}
\newcommand{\Op}{\partial^{\bigoplus}}
\newcommand{\op}[1]{\partial^{#1\bigoplus}}
\DeclareMathOperator{\interior}{int}

\usepackage{listings}
\usepackage{xcolor}
\lstset { %
    language=C++,
    backgroundcolor=\color{black!5}, % set backgroundcolor
    basicstyle=\footnotesize,% basic font setting
}

\newtheorem{definicija}{Definition}[section]
\newtheorem{trditev}{Claim}[section]
\newtheorem{izrek}{Theorem}[section]
\newtheorem{opomba}{Remark}[section]
\newtheorem{corollary}{Corollary}[section]
\newtheorem{proposition}{Proposition}[section]

\usepackage[hidelinks]{hyperref}

\title{$\DD^n\CC$}
\author{Žiga Sajovic}

\begin{document}

\maketitle

\begin{abstract}
We provide an illustrative implementation of an analytic, infinitely-differentiable machine, implementing infinitely-differentiable programming spaces and operators acting upon them, constructed in the paper \emph{Operational calculus on programming spaces and generalized tensor networks}. Implementation closely follows theorems and derivations of the paper, intended as an educational guide.
\end{abstract}

\tableofcontents

\section{Introduction}

We provide an illustrative implementation of the virtual memory $\VV$ and its expansion to $\VV\otimes T(\VV^*)$, serving by itself as an algebra of programs and an infinitely differentiable programming space
\begin{equation}
\DD^n\CC<\dP_n:\VV\to\VV\otimes T(\VV^*)
\end{equation}
acting on it.

We provide a construction of operators
\begin{equation}
\DD^n=\{\D^k;\quad 0\le k\le n\}
\end{equation}
allowing for implementation of the operator
\begin{equation}\label{eq:sumd}
\sumd_n=1+\D+\D^2+\cdots+\D^n
\end{equation}
increasing the order of a differentiable programming space
\begin{equation}
\sumd_n:\dP_k\to\dP_{n+k}
\end{equation}

All required theorems and proofs are provided by the paper \emph{Operational calculus on programming spaces and generalized tensor networks}. Source code can be found on github \cite{dCpp}.

\section{Virtual memory}

We model an element of the virtual memory $v\in\VV_n$ 
\begin{eqnarray}
\VV_{n}=\VV_{n-1}\oplus(V_{n-1}\otimes\VV^*) \\
\VV_{n}=\VV\oplus\VV\otimes\VV^*\oplus\cdots\oplus\VV\otimes\VV^{*n\otimes}\\ \label{eq:V_n}
\VV_n=\VV\otimes T(\VV^*)
\end{eqnarray}
with the class $var$.

\begin{lstlisting}
template<class V>
class var
{
    public:
    	int order;
        V id;
        std::map<&var,var>* dTau;

        var();
        var(const var& other);
        ~var();
        var(double n);
        void init();
        /*
        *declerations of algebraic operations
        */
};
\end{lstlisting}

The expanded virtual memory $\VV_n$ is the tensor product of the virtual memory $\VV$ with the tensor algebra of its dual. The address \emph{\&var} stands for the component of $v\in\VV_{n-1}$ the tensor product with the component of $v^*\in\VV^*$ was computed on to generate $v\in\VV_n$ in equation \eqref{eq:V_n}. This depth is contained in the \emph{int order}.

Tensor products of the virtual memory $\VV$ with its dual are modeled using maps, denoted by \emph{dTau}. Naming reflects how the algebra will be constructed, mimicking the operator $\sumd_nn$ \eqref{eq:sumd}.

\subsection{Algebra over a field}

 Algebra over a field is a vector space equipped with a bilinear product. Thus, an algebra is an algebraic structure, which consists of a set, together with operations of multiplication, addition, and scalar multiplication by elements of the underlying field.
 
 Algebra is constructed by mimicking a direct sum of operators $\sumd_n$ \eqref{eq:sumd} mapping $\dP_0\oplus\dP_0\to\dP_n$. This is reflected in the structure of the class $var$  modeling the elements of the virtual space $v\in\VV_n$. Compositions are modeled by mimicking the projection of the operator $\exp(\D_fe^{h\D_g})$, generalizing both forward and reverse mode automatic differentiation, to the unitary ball and applying it to the resulting direct sum.
 
\begin{izrek}
An instance of the class $var$ is an element of the virtual memory $\VV_n$.
 \begin{equation}
 var\in\VV_n
 \end{equation}
\end{izrek}
\begin{proof}
\begin{equation}
 id\in \VV\land  dTau\in \VV_{n-1}
 \end{equation}
 $$\land$$
 \begin{equation}
 var=id\oplus dTau\implies var\in\VV_n
 \end{equation}


\end{proof}

\subsubsection{Vector space over a field $K$}
 
To firstly construct a vector space $\VV_n$ over a field $K$, we implement scalar multiplication and addition. 

We begin with scalar multiplication.

\begin{lstlisting}
template<class K>
var var::operator*(K n)const{
		var out;
        out.id=this->id*n;
        for_each_copy(...,mul_make_pair<pair<&var,var> >, n);
        return out;
}

template<class K>
var var::operator/(K n)const {...};
\end{lstlisting}
Scalar multiplication and its convenient inverse employ the function \emph{for\_each\_copy}, applying the provided operation 
\begin{lstlisting}
template<class V, class K>
V mul_make_pair(V v, K n) {
  return std::make_pair(v.first, v.second * n);
}
\end{lstlisting}
to each one of the components of $this$ and storing the result in $out.dTau$.

Vector addition by component is implemented by 
\begin{lstlisting}
var var::operator+(const var& v)const{
		var out;
        out.id=this->id+v.id;
        merge_apply(..., sum_pairs<pair<int, double> >);
            return out;
}
\end{lstlisting}
Vector addition by component employs the function \emph{merge\_apply}, applying the provided function \emph{sum\_pairs}
\begin{lstlisting}
template<class V>
T sum_pairs(V v1, V v2) {
  return std::make_pair(v1.first, v1.second + v2.second);
}
\end{lstlisting}
to corresponding components, storing the result in $out.dTau$, in $\mathcal{O}(n\log(n))$.

\begin{izrek}
Class $var$ models a vector space over a field $K$.
\end{izrek}
\begin{proof}
By implementations of addition by components and multiplication with a scalar $K$, the axioms of the vector space are satisfied.
\end{proof}

\subsubsection{Algebra over a field $K$}

With the vector space constructed, we turn towards constructing the algebra. To construct an algebra over a field $K$, we equip the vector space $\VV_n$ with a bilinear product by components.

\begin{lstlisting}
var var::operator*(const var& v)const{
		var out;
        out.id=this->real*v.id;
        if(max(v.order,this->order)>0){
        	map<int,double> tmp1;
        	map<int,double> tmp2;
        	for_each_copy(...,mul_make_pair<pair<&var,var> >, 
        		v.reduceOrder());
        	for_each_copy(...,mul_make_pair<pair<&var,var> >, 
        		this->reduceOrder());
        	merge_apply(..., sum_pairs<pair<&var,var> >);
        }
        return out;
}
\end{lstlisting}
The employed functions have been explained at previously usage. The \emph{reduce.Order} function makes a shallow copy of $this$, while reducing the order of the returned copy. This bilinear product contains Leibniz rule within its structure.

\begin{izrek}
Class $var$ models an algebra over a field $K$.
\end{izrek}
\begin{proof}
By the implementation of a bilinear product by components the axioms of an algebra over a field are satisfied.
\end{proof}

For ease of expression we implement exponentiation,

\begin{lstlisting}
var var::operator^(double n) const{
		var out;
        out.id=std::pow(this->real,n);
        if(this->order>0){
        	for_each_copy(...,powTimes<pair<&var,var> >, 
        		this->reduceOrder(),n);
        }
        return out;
}
\end{lstlisting}
employing the function \emph{for\_each\_copy}, applying the provided operation 
\begin{lstlisting}
template<class T, class V, class K>
T powTimes(T v1, V v2, K n) {
  return std::make_pair(v1.first, n*(v2^(n-1))*v1.second);
}
\end{lstlisting}
to each component.

The above is sufficient to implement the following decelerations.
\begin{lstlisting}
var operator*(double n)const;
var operator+(double n)const;
var operator-(double n)const;
var operator/(double n)const;
var operator*(const var& v)const;
var operator/(const var& v)const;
var operator+(const var& v)const;
var operator-(const var& v)const;
var operator^(double n) const;
var operator-()const;
var& operator=(const var& v);
var& operator=(double n);
var& operator+=(const var& v);
var& operator-=(const var& v);
var& operator*=(const var& v);
var& operator/=(const var& v);
var& operator*=(double n);
var& operator/=(double n);
var& operator+=(double n);
var& operator-=(double n);
var operator*(double n, const var& v);
var operator+(double n, const var& v);
var operator-(double n, const var& v);
var operator/(double n, const var& v);
var operator^(double n, const var& v);
\end{lstlisting} 

\section{Analytic virtual machine}

\begin{definicija}[Analytic virtual machine]
The tuple $M=\langle \VV,\dP_0\rangle$ is an analytic, infinitely  differentiable virtual machine.
   
    \begin{itemize}
    \item
    $\VV$ is a finite dimensional vector space
    \item
    $\VV\otimes T(\VV^*)$ is the virtual memory space, serving as alphabet symbols
    \item
    $\dP_0$ is an analytic programming space over $\VV$
    \end{itemize}
    When $\dP_0$ is a differentiable programming space, this defines an
    infinitely differentiable virtual machine.
  \end{definicija}
  
  \subsection{Operators}
        
        The operator 
        \begin{equation}\label{eq:DD}
          	\sumd_n = 1+\D +\D^2 +\ldots + \D^n 
          \end{equation}
        is used to implement $\DD^n\CC\subset\dP_n$, employing the recursive relation
        \begin{equation}
              \label{eq:rekurzija}
              \sumd_{k+1} = 1+\D\sumd_k.
            \end{equation}
        \begin{equation}
        \sumd_{k+1}\CC = \CC+\D\sumd_k\CC
        \end{equation}
        
        The \emph{double id} stands for the identity operator $1$ in the expression and the \emph{map<\&var, var> dTau} for the operator $\D\sumd_k$.
        
        Compositions of these operators is achieved by mimicking the generalized pullback operator
        \begin{equation}\label{eq:opKompo}
          \exp(\D_fe^{h\D_g})(g): \dP\to\dP_\infty(g)
          \end{equation}
          projected onto the unit sphere.
          
          \begin{opomba}
          Implementation in this paper is intended to be simply understood and is written as such. But, with the existing algebra and operational calculus, one could easily implement the operator $\exp(\D_fe^{h\D_g})(g)$ by any of the efficient techniques available, as it is given by a generating function.
          \end{opomba}
        
        \begin{lstlisting}
        typedef var (*dTau)(var);
        
        template<class K>
        class tau
        {
            public:
                tau();
                tau(K mapping, var dTau);
                ~tau();
                var operator()(const var&v);
            private:
                dTau primitive;
                K mapping;
        };
        \end{lstlisting}
        
        \begin{lstlisting}
        var tau::operator()(const var&v){
            var out;
            out.id=mapping(v.id);
            for_each_copy(...,mul_make_pair<std::pair<&var,var> >, 
            	primitive(v));
            return out;
        }
        \end{lstlisting}
        
          \subsection{Differentiable programming space}
              
              With the algebra over $\VV_n$ implemented, we turn to the construction of a differentiable programming space
              
              \begin{equation}
              \CC:\VV\to\VV
              \end{equation}
              
              \begin{definicija}\label{def:dP}
               	A \emph{differentiable programming space} $\dP_0$ is any subspace of $\F_0$ such that
               	\begin{equation}\label{eq:P}
               	\D\dP_0\subset\dP_0\otimes T(V^*)
               	\end{equation}
               	When all elements of $\dP_0$ are analytic, we denote $\dP_0$ as an \emph{analytic programming space}.
               \end{definicija}
              
              \begin{izrek}\label{izr:P}
              	Any differentiable programming space $\dP_0$ is an
                infinitely differentiable programming space, such that
              	\begin{equation}\label{eq:P_n}
              	 		\D^k\dP_0\subset\dP_0\otimes T(V^*)
              	 	\end{equation}
              for any $k\in\mathbb{N}$.
              \end{izrek}
              
              Thus, in order to have a differential programming space, we must provide closure under the differential operator, for the function space $\CC$, expanded by the tensor product with the tensor algebra of the dual of the virtual memory $\VV$.
              
              \begin{equation}
              \DD^n\CC<\dP_n\iff\DD^n\CC\subset\CC\otimes T(\VV^*)
              \end{equation}

\begin{trditev}
Any library implementing functions acting on variables of type $double$ or $float$, could be trivially included into $\CC<\dP_0$, simply by replacing all variables with the class $var$. Moreover, any implementations using the implemented algebra in its construction of functions, are contained in $\dP_0$.
\end{trditev}

\subsubsection{Example}
  
As an illustrative example, we provide a simple implementation of a differentiable programming space through the use of the operator $tau$. Assume the existence of functions
\begin{equation}
\sin\_double:double\to double
\end{equation}
\begin{equation}
\cos\_double:double\to double
\end{equation} 
\begin{equation}
e\_double:double\to double
\end{equation}  
\begin{equation}
\ln\_double:double\to double
\end{equation} 
filling the set spanning $\CC$.

Previous declarations of needed functions are assumed. 

\begin{lstlisting}
namespace dCpp{
var sin(const var& v);
tau cos;
tau e;
tau ln;
var cos_primitive(const var& v);
var ln_primitive(const var& v);
var e_primitive(const var& v);
}
\end{lstlisting}

We construct the map $sin$ explicitly for educational purposes

\begin{lstlisting}
var dC::sin(const var& v){
    var out;
    out.id=std::sin_double(v.id);
    if(v.order>0){
    	for_each_copy(...,mul_make_pair<std::pair<&var,var> >,
    		cos(v.reduceOrder()));
    }
    return out;
}
\end{lstlisting}

Other maps are constructed through employment of the operator $tau$.

\begin{lstlisting}
typedef double (*mapping)(double);

dCpp::cos_primitive(cont $var v){
	return (-1)*sin(v.reduceOrder());
}

dCpp::cos=tau<mapping>(cos_double,cos_primitive);

dCpp::ln_primitive(cont $var v){
	return 1/v.reduceOrder();
}

dCpp::ln=tau<mapping>(ln_double,ln_primitive);

dCpp::e_primitive(cont $var v){
	return e(v.reduceOrder());
}

tau dCpp::e=tau<mapping>(e_double,e_primitive);

\end{lstlisting}

\begin{izrek}\label{izr:dCpp}
The programming space $dCpp$ is a differentiable programming space satisfying
\begin{equation}
dCpp<\dP_0\iff \DD dCpp\subset dCpp\otimes T(\VV^*)
\end{equation}
\end{izrek}
\begin{proof}
\begin{equation}
\forall_{\phi_i\in dCpp}\exists_{\phi_j\in dCpp}(\phi_i.primitive=\phi_j)
\end{equation}
$$\implies$$
\begin{equation}
\DD dCpp\subset dCpp\otimes T(\VV^*)
\end{equation}
\end{proof}
\begin{corollary}
The programming space $dCpp$ is an infinitely-differentiable programming space satisfying
\begin{equation}
dCpp<\dP_0\iff \DD^n dCpp\subset dCpp\otimes T(\VV^*)
\end{equation}
\end{corollary}
\begin{proof}
Follows directly from Theorem \ref{izr:dCpp} by Theorem \ref{izr:P}.
\end{proof}

\subsection{External libraries}

Any $\CC$ library written in the generic paradigm employing templates is fully compatible with differentiable programming space and the virtual memory $\VV$. 

We illustrate on the example of Eigen. We will code a perceptron with sigmoid activations, followed by softmax normalization, taking 28x28 image as an input and outputting a 10 class classifier. Existence of needed, but trivial and intuitively understood mappings is assumed (ex. $init$ initializes elements to the desired order). 

\begin{lstlisting}
template <typename Derived>
    void softmax(Eigen::MatrixBase<Derived>& matrix){
            //maps each element of the matrix by y=e^x;
            dCpp::map_by_element(matrix,&dCpp::e);
            //sums the elements of the matrix using Eigens function
            var tmp=matrix.sum();
            //divides each element by the sum
            for (size_t i=0, nRows=matrix.rows(), 
            	nCols=matrix.cols(); i<nCols; ++i)
                for (size_t j=0; j<nRows; ++j){
                    matrix(j,i)/=tmp;
                }
}

int main(){
//order of derivatives needed
int order=...;
//    Matrix holding the inputs (imgSizeX1 vector)
const int imgSize=28*28;
const Eigen::Matrix<var,1,imgSize>input=Eigen::Matrix<var,1,
	imgSize>::Random(1,imgSize);
//    number of outputs of the layer
const int numOfOutOnFirstLevel=10;
//    matrix of weights on the first level 
//    (imgSizeXnumOfOutOnFirstLevel)
Eigen::Matrix<var,imgSize,numOfOutOnFirstLevel>firstLayerVars=
	Eigen::Matrix<var,imgSize,numOfOutOnFirstLevel>::
	Random(imgSize,numOfOutOnFirstLevel);
//    initializing weights
dCpp::init(firstLayerVars, order);
//    mapping of the first layer --> resulting in 10x1 vector
Eigen::Matrix<var,numOfOutOnFirstLevel,1>firstLayerOutput=
	input*firstLayerVars;
//    apply sigmoid layer --> resulting in 10x1 vector
dCpp::map_by_element(firstLayerOutput,&dCpp::sigmoid);
//    apply sofmax layer --> resulting in 10x1 vector
softmax(firstLayerOutput);
//retrieve the computed derivatives
\end{lstlisting}

  \printbibliography
  
\end{document}